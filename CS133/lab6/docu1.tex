\documentclass{IEEEtran}

% Preamble
\title{An Introduction to BibTeX}
\author{Aaron Baw}
\date{\today}
\usepackage{algorithmic}
\usepackage{algorithm}
\usepackage{listings}
% Preamble end

\begin{document}

% Docu begin
\maketitle{An introduction to BibTeX}


It is required of society to reference the works of H.G. Wellington-Boots \cite{label1} to understand the need for a type of
requirements that should be consulted when referring to the absolute definitions of a particular encounter.

Another important detail to keep in mind is that of the world renowned scientists \cite{label2, label3} that have already explored such matters in great depth.


\section{Using Algorithms}

Algorithms have their roots in the most fundamental forms of maths. Below I shall demonstrate one of the most useful equations that
humans tend to abide by.
\begin{algorithm}
  \begin{algorithmic}
    \STATE $n=10$
    \FOR{i=0; $i<=n$; i++} \STATE{Loop Through for.}
    \IF{n Even} \PRINT i \ENDIF \ENDFOR
  \end{algorithmic}
  \caption{A Loop which prints out numbers from 0 to n.}
  \label{for1}
\end{algorithm}

\lstset{language=Java}
\begin{lstlisting}
String word = "Apple";
for (int i = 0; i < word.length; i++){
  if (1%2==0) System.out.println(word[i]);
}
\end{lstlisting}


\bibliographystyle{ieeetr}
\bibliography{bibliography}
\end{document}
