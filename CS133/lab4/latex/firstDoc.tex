\documentclass[12pt, twocolumn]{book} %Set a new doc with two columns, 12pts.

% Setting up the preamble.
\title{My First LaTeX Document}
\author{Aaron Baw}
\usepackage{graphicx}
\date{\today}
% END of preamble.

% Document start
\begin{document}

\maketitle % This prints the title on top of the document.

\tableofcontents


\chapter{The Beginning, of the beginning.}
Here you will find possibly the most confusing and pointless piece of sub-section limbo imagineable.

\section{The First section, of the beginning, of the beginning.}

\subsection{The first sub-section of the first section of the beginning of the beginning.}

\subsubsection{The first sub-sub-section of the first sub-section of the first section of the beginning of the beginning}

Obi-Wan and Luke were at a bar. They ordered the giner-ale and samosa. Then he said to Luke:

  ``Use the sauce Luke!''

And so he did.

One more interesting thing to note is that lorem ipsum is completely dummy.

\chapter{Enumeration practice}

Things I am good at:
\begin{enumerate}
\item Appreciating the Geese
\item Pondering life and the Universe
\item Procrastinating in productive ways
\item finishing lis
\end{enumerate}


Other things that are worth noting about me:
\begin{enumerate}
\item I wasn't sure what to write here
\item Read above
\item Almost done
\item I have completed my list!
\end{enumerate}

\chapter{Adding Graphics}
\section{Admiring Nigel}
\begin{figure}[h]
  \begin{center}
    \includegraphics[width=120mm]{/dcs/16/u1617781/Pictures/Ukip-leader-Nigel-Farage.jpg}
    \caption{Just look at Nigel. Isn't he a happy fellow?}
    \label{reflabel}
  \end{center}
\end{figure}

Politicians are rich sources of ridiculous faces. Observe the next president of the free world.

\section{Over-excited politicians}
\begin{figure}[h]
  \begin{center}
    \includegraphics[width=100mm]{/dcs/16/u1617781/Pictures/hillary_balloons.jpg}
    \caption{I think this just speaks for itself really.}
    \label{reflabel}
  \end{center}
\end{figure}


\chapter{Using Mathematical Symbols}

\begin{equation}
  x = \frac{\pm}
\end{equation}

\end{document}
